%*****************************************
\chapter{Diskussion}\label{ch:discussion}
%*****************************************
In der Diskussion werden die Ergebnisse der Arbeit kritisch bewertet.
So wäre hier z.B. darzulegen, dass man zwar eine schöne Modellierungsmethode zur Beschreibung der Kommunikation im Krankenhausinformationssystem gefunden hat, diese aber so aufwändig ist, dass kaum jemand sie benutzen wird, wenn nicht folgende weiteren Arbeiten noch durchgeführt werden.... So wird die Diskussion dann auch einen Ausblick auf die Dinge enthalten, die eigentlich noch zu erledigen sind.
Gerade in einer Seminararbeit könnte hier auch die Kritik des Autors an dem stehen, was er in der Literatur zu dem zu bearbeitenden Thema hier und da gelesen hat
\graffito{Dies ist eine Notiz}


\newpage
Grundlage für eine Abschlussarbeit ist eine gründliche Recherche im Themenumfeld. Dabei ist es ausdrücklich nicht hinreichend, mit bekannten Suchmaschinen im Internet zu recherchieren. Vielmehr wird von den Studierenden erwartet, dass sie auch referierte Veröffentlichungen (wissenschaftliche Zeitschriften (auch elektronisch), Bücher) in die Erarbeitung einbeziehen (und mit entsprechenden Quellenangaben belegen). Informationen zum Thema der Literaturrecherche finden sich in den \href{http://www.imise.uni-leipzig.de/Lehre/MedInf/Abschlussarbeiten/Literaturrecherche.jsp}{Hinweisen zur Literaturrecherche}.
\paragraph{Beispielzitierungen}
\citet{sniktec} beschreiben ein Verfahren zur X von Y auf Basis von Z.
Alternativ: X von Y lässt sich auf Basis von Z ermitteln~\citep{sniktec}.