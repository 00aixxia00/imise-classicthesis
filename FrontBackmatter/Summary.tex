%*******************************************************
% Summary
%*******************************************************
\pdfbookmark[1]{Zusammenfassung}{summary}
\chapter*{Zusammenfassung}
\addcontentsline{toc}{chapter}{Zusammenfassung}
In der Zusammenfassung wird vor allem zusammenfassend dargestellt, welche Ergebnisse zu den formulierten Zielen erreicht wurden. Wurden Fragen formuliert, können auch diese hier beantwortet werden. So soll es möglich sein, dass ein Leser von der Arbeit lediglich die Einleitung und die Zusammenfassung lesen und doch die Ergebnisse der Arbeit erfassen kann.

Zusammenfassung der Arbeit. Der Abschnitt dient dem Leser zur schnellen Orientierung und Einordnung der vorliegenden Arbeit. Bei der Formulierung sollten wichtige Schlüsselbegriffe verwendet werden.

Checkliste:
\begin{enumerate}
\item Kurze, klare Darstellung zu Gegenstand, Fragestellung und Zielsetzung der Arbeit
\item Angewandte Methodik
\item Vorstellung der wesentlichen Ergebnisse und daraus resultierende Schlussfolgerungen.
\end{enumerate}
Für wissenschaftliche Abschlussarbeiten sind etwa 200 bis 250 Wörter ausreichend.


\vfill
